%-------------------------------------------------------------
\chapter{Conclusion}

First we summarize the steps that we have followed in our CODE and 
then the results concerning the spatial orientation of galaxies in three
rotating clusters (A2162, A1139, A2366) that we obtained by using CODE will be discussed.
The code named GOBAD1.0 follows the following steps:
\begin{itemize}
\item It propagates number statistics in order to find appropriate bin size for given parameters. 
\item It makes input files by creating 1$0^n$ random virtual galaxies (Aryal \& Saurer 2000) for the given parameters and it gives expected polar and azimuthal angle distributions following Godlowskian method (Flin \& Godlowski 1986).
\item It calculates observed distribution by using database of galaxies that we obtained from Huyang \& Lee (2007)
\item A comparison between the observed and expected distribution can me made by using our CODE, using various statistical tests, namely $\chi^2$-test, auto-correlation test, Fourier test, K-S test and K-V test. 
\end{itemize}
We have studied spatial orientations of angular momentum vectors
of galaxies in three rotating clusters (A2162, A1139, A2366) that are in dynamical
equilibrium with a single number-density peak around the cluster
center and a spatial segregation of high- and low-velocity
galaxies. We conclude our results as follows:

\begin{itemize}
\item A random orientation is noticed for all three rotating clusters
supporting hierarchy model as predicted by Peeples (1969). The
clusters that are rotating and in the dynamical equilibrium showed
a random orientation of spin vector and spin vector projections
with respect to both galactic and Supergalactic coordinate
systems. As expected the choice of coordinate system is found to
be independent of the spatial orientation of the galaxies in the
cluster.

\item We noticed that the vanishing angular momentum is
preferred by the rotating clusters that have larger value of
velocity dispersion. The existence of substructures is probably a
signature of non-vanishing angular momenta. Thus, we found a good
correlation between hierarchy model (Peebles 1969) and the Li's
model (Li 1998) concerning the evolution of large scale
structures. 

%\item Godlowski (2011) discussed the observational aspect of
%rotations in the universe on different scales and concluded by
%advocating Li's (1998) model in which galaxies are believed to be
%formed in the rotating universe.  However the results of Aryal
%\& Saurer (2006) as well as those of Godlowski et al. (2010) are
%consistent  with the predictions of Li's model, they may be as
%well interpreted as the effect of tidal forces, according to the
%scenario of Catelan \& Theuns (1996). A  similar results are
%obtained by Aryal et al. (2007) and Godlowski et al. (2005). 
%In the hierarchical clustering model, the tidal torque scenario
%naturally arises and hence the distribution of spin angular
%momenta of galaxies become random. However a local tidal shear
%tensor can cause a local alignment of rotational axes (Dubinski
%1992, Lee \& Pen 2000, Lee 2004, Trujillo et al. 2006). 

\end{itemize}
\section{Future Works}

Following works are recommended for the future:
\begin{itemize}
 \item In the future we intend to study preferred alignments in the rotating
cluster candidates that are not in the dynamical equilibrium.
\item In future we intend to define a physical reference plane to study the distribution of the spin vector orinetation.

\end{itemize}