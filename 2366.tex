Struble and Rood (1987) presented list of redshifts  for 578 Abell clusters, and velocity dispersion for 77, available as of 1986 July. The redshifts  is 0.542 and in this 1 galaxy was considered. The velocity dispersion is 14 km\,s$^{-1}$.\\\\
Lambas, Groth and Peebles (1988) confirmed that Argyres et al. (1986) result that galaxy count on scale to at least 15 $h^{-1}$Mpc are systematically higher in the direction of the major axes of brightest cluster galaxies using independent sample of brightest galaxies in rich clusters in the southern galactic hemisphere. The abell 2366 has redshift of 0.054 and the postion angle of $115^\circ$.\\\\
Schombert (1988) used single-band photographic and CCD surface photometry to determine luminosity and structural properties of the extended, faint envelopes around 27 cD galaxies. Galaxies classified as cD occupy a special place in the scheme of extragalactic structure by being intermediate in scale between normal galaxies and cluster-sized entities and may be important for tracing the behavior of cluster dynamics (Tonry 1987). Abell 2366 is BM I-III and luminosity of G1 (F) galaxy is log($M_{gal}$)=11.02 and the luminosity of the envelope of this galaxy is log(L$_{env}$)=10.52.\\\\
Baier and Ziener (1989)  presented the compilation of publication with photometeric data in and around clusters of galaxies. Abell 2366 was studied with photographic data by Malumuth (1985), Horbey (1983), Murphy (1983), Schombert (1986), Struble (1987).\\\\
West (1989) generated a catlog of 48 probable supercluster by means of percolation technique using a large sample of the Abell clusters with measured redshifts $z\leq$0.1 and various properties of these supercluster are then examined. They also determined the orientation of clusters within these supercluster. Position angle data for cluster were gathered from many sources. Abell 2366 lies on the supercluster number 19. The position angle is 161 (Struble and Peeble 1985), 115 (Lambas, Groth and Peebles 1988), 143 (Struble 1987, position angle is determined for various surface brightness isophotes of first-ranked galaxies). The average of the position angle is 139$^\circ$  with standard deviation 19$^\circ$.\\\\
Abell, Corwin and Olowin (1989) compiled a all-sky catalog of 4073 rich clusters of galaxies, each having at least 30 members within the magnitude range $m_3$ to $m_3+2$ (m$_3$ is the magnitude of the third brightest cluster member) and each with a nominal redshift less than 0.2. Abell 2366 is BM type I-II cluster (Bautz and Morgan 1970) and the background corrected count of cluster member in magnitude range $m_3$ to $m_3$+2 is 47. The cluster redshift (Struble and Rood, 1987) is 0.0542. The richness and distance class is 0 and 4 respectively (Abell 1958). $m_{10}$ (the red magnitude of tenth brightest cluster ) is 15.9 (Abell 1958).\\\\
Sandage and Perelmuter (1981) obtained petrosian radii, effective radii, apparent magnitude and surface brightness of few ranked galaxies in 56 nearby cluster and groups using data from literature. A2366 has redshift 0.0542 and factor f is 1.577. It is  the factor by which to mutiply the angular radius to calculate the linear radius in kpc. The factor f is derived from the redshift and an assumed Hubble constant of 50 km\,s$^{-1}$Mpc$^{-1}$. The total V magnitude from the calculated and corrected $m(\infty)$ growth curve is 13.4 and adopted distance radius from redshifts is 37.56. The absolute magnitude is -24.2. The measured angular ``effective " radius in arcsecond is $27.5^{\prime\prime}$ and the resulting linear radius obtaining by multiplying $r_e^{\prime\prime}$ by f is 15.4 and the effective surface brightness is  19.8.\\\\
Plionis, Barrow and Frenk (1991) identified a large number of galaxy cluster in the Lick map using algorithm based on an overdensity criterion. The resulting catlogues contain $\sim$ 6000 clusters (with $|b|\geq40^\circ$) out of which 753 are Abell clusters. In this paper they were mainly concerned with the shapes of galaxy clusters found in projection and used the Lick map of galaxies as their basic data set. The  mean ellipticity of Abell cluster A2366 (identified at 3.6 level) is 0.193.\\\\
Postman, Huchra and Geller (1992) acquired redshifts for a complete sample of 351 Abell clusters with tenth ranked galaxy magnitude ($m_{10}$) less than or equal to 16.5. Abell 2366 has richness and distance class 0 and  4 respecively. The first ranked magnitudes and tenth-ranked magnitude are 13.20 and 15.90 respectively. The estimated redshift (Leir \& Bergh 1977) is 0.0600 and observed mean redshift 0.0545 using 2 galaxies.\\\\
Plionis (1994) estimated position angles of a large number of Abell and Schetman clusters, identified in the Lick map as surface galaxy-density enhancements. Abell 2366 also known as Schetman 429 has postion angle 133.6.\\\\
Cox, Bregman and Schombert (1995) examined the frequency with which central dominant galaxies are sources of far-infrared emission in a complete sample of galaxies. But for Abell 2366 they didn't detect anything.\\\\
Struble and Rood (1999) presented a compilation of redshifts for 1572 Abell, Corwin \& Olowin (ACO) clusters, referenced to both the heliocentric and cosmic background radiation reference frames. Abell 2366 which was already listed in Abell (1958) has adopted heliocentric redshifts $(\bar{z}_h)$ 0.0529 and redshift with respect to cosmic background radiation ($\bar{z}_c$) is 0.0517 which is derived from $\bar{z}_h$. The adopted velocity disperion in rest frame of cluster $\sigma_{corr}$ is 490 km\,s$^{-1}$.