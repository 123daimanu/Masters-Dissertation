%\thispagestyle{plain}
%\chaptermark{Abstract}
%\addcontentsline{toc}{chapter}{Abstract}
% \noindent


\begin{abstract}
{We present a new CODE for the study of spatial orientation of spin vectors of galaxies that uses
several steps, particularly for the data reduction, coordinate transformation, numerical simulation
and statistical tests in order to find expected isotropic and observed distributions. This CODE is
written in GNU Octave, version 3.2.4. GNU Octave is open source programming language which we have used in Ubuntu 12.04.
We have used this CODE (named GOBAD1.0) to study the spatial orientations of angular momentum vectors of
galaxies in three rotating clusters (A1139, A2162,
and A2366) that show a single number-density peak around
the cluster center with a spatial segregation of high- and
low-velocity galaxies. Li model explains the formation and evolution of large structure by considering the global rotation of the universe. It correctly predits the empirical relation between angular momentum and mass of the large structure and explains the vanishing angular momentum of some intermediate mass of structures. 
The positions, position angles and
inclination angles are used to covert two-dimensional observed
parameters into three-dimensional angular momentum vectors of the
galaxy. The expected isotropic distribution curves are determined
by running numerical simulations. The chi-square, autocorrelation,
Fourier, K-S, Kuiper-V tests are carried out in order to examine
non-random effects in the expected isotropic distributions. In
general a random orientation of angular momentum vectors of
galaxies are found in all three rotating clusters supporting
Hierarchy model of galaxy formation. It is found that the
vanishing angular momenta is preferred by the rotating clusters
which are in dynamical equilibrium and have larger value of
velocity dispersion.}
\end{abstract}
