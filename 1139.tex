Sun et al. (2008) presented a systematic analysis of 43 nearby galaxy groups (kT$_{500}=0.7-2.7$ keV or M$_{500}=10^{13}-10^{14}h^{-1}$M$_{\odot}$, $0.012<z<0.12$), based on Chandra archival data. A1139 is the faintest in 2-3 keV system. The redshift extracted from NASA/IPAC Extragalactic Database is 0.0398 and the luminosity distance derived from this redshift is 169. 2MASS $K_s$ band luminosity in terms of log($L_{Ks}/L_{\odot}$) is 11.50 and 1.4 GHz luminosity in log($L_{1.4\text{GHz}}${W Hz}$^{-1}$) from the NRAO VLA Sky Survey (NVSS) or the Sydney University Molonglo Sky Survey (SUMSS), assuming a spectral index of -0.8 is 23.16. Temperature of the hotter component of local soft CXB is 0.25 and the observed flux of the local soft CXB (in unit of $10^{-12}$ ergs cm$^{-2}$\,s$^{-1}$\,deg$^{-2}$) in 0.47-1.21\,keV is $2.6^{+0.4}_{-0.5}$.\\\\
Holden et al. (2009)  compiled a sample of early type cluster galaxies from $0<z<1.3$ and measured the evolution of their ellipticity distribution. For $z<0.05$ sample is selected from Abell cluster in SDSS fifth Data Release (McCarty et al. 2007, SDSS DR5). The redshift of Abell 1139 is 0.0383 (Struble \& Rood 1999). They calculated $R_{200}$ using the formula given in Calberg et al. 1997. In Abell 1139 the velocity dispersion is 436 km\,s$^{-1}$ (Poggianti et al. 2006). The calculated value of $R_{200}$ is 0.67 Mpc and found the number of early type galaxies within $R_{200}$ to be 23.\\\\
Dong, Rasmussen and Mulchaey (2010) performed search for X-ray cavities in hot gas of 51 galaxy groups with Chandra archival data. The cavities were identified based on two method: subtracting an elliptical $\beta$- model fitted to the X-ray brightness, and performing unsharp masking. They found tight correlations between the radial and tangential radii of the cavities and between their size and projected distance from group center, in quantitative agreement with the case for more massive clusters. They suggests that similar physical process are responsible for cavity evolution and disruption in systems covering a large range in total mass. Abell 1139  has no detectable cavities and number of 0.3-2 keV photons from diffuse emission on the central CCD is 1948. The radio luminosity at 1.4 GHz of any radio source within the central bright group galaxy, extracted from the NRAO VLA Sky Survey (Condon et al. 1998) is less than 22.09 W\,Hz$^{-1}$.\\\\
Wojtak and Lokas (2010) analyzed kinematic data of 41 nearby ($z<0.1$) relaxed galaxy clusters in terms of the projected phase-space density using phenomenological, fully anisotriopic model of distribution function. They apply Markov Chain Monte Carlo approach to place constraints on the total mass distribution approximated by the universal NFW profile and the profile of the anisotrypy of galaxy orbits. The  Abell 1139 is BM III type cluster and no thing can be said about the cool core as judged on the basis of Hudson et al. (2010), Chen et al. (2007), White (2000), Jones and Forman (1997). The mass and radius parameter used in their analysis of Abell 1139 is $0.41^{+0.12}_{-0.05}10^{14}$ M$_{\odot}$ and $0.39^{+0.13}_{-0.08}$ Mpc. The viralized mass of this cluster is $4.10^{+0.39}_{-0.36}$ and the concentration parameter is $3.53^{+0.51}_{-1.25}$. The ratio of the radial velocity dispersion and the tangential velocity dispersion at the cluster center and viral radius is $0.87^{+.28}_{-0.19}$ and $1.75^{+0.22}_{-0.28}$ respectively. The anisotropy at the cluster center and virial radius is $-0.31^{+0.56}_{-0.83}$ and $0.67^{+0.07}_{-0.14}$ respectively. In their study the virial mass of the clusters correlates very well with the velocity dispersion, the X-ray temperature and the X-ray luminosity of the clusters.\\\\
Harrison et al. (2010) presented a catlogue of galaxies well suited to the investigation of the early type galaxy formation and evoluiton. A1139 has velocity dispersion of $504\pm 47$ km\,s$^{-1}$ and the total number of galaxies used to calculate the red shift of $0.0396\pm 0.0002$ is 106 and it is B-M III cluster.\\\\
Poggianti et al. (2004) studied how the proportion of star-forming galaxies evolves as a function of galaxy environment, using the [$O_{II}$] line in emission as a signature of ongoing star formaion. In their study they used 10 galaxies of Abell 1139 to find its [$O_{II}$] fraction. They also calculated $R_{200}$ to be 1.06 Mpc. The value of [$O_{II}$] fraction corrected for completeness and uncorrected for completeness are $0.38\pm0.20$ and 0.40 respectively.\\\\
Popesso et al. (2008) considerd a large sample of optically selected clusters, in order to elucidate the physical reason for the existence of X-ray underluminous clusters. For this purpose they analyze the correlation of X-ray and optical properties of a sample of 137 spectroscopically confirmed abell clusters in SDSS database. They used the properties extracted from the SDSS DR3 paper. According to this paper, Abell 1139 has 89 cluster members within 1 Abell radius. The cluster red shift is 0.0395 and the velocity dispersion is 376$\pm$34. The mass of the cluster within $r_{200}$ is 1.68 $\times 10^{14}$M$_{\odot}$. The virial radius of the cluster is 1.2 Mpc. Optical luminosity of the cluster is (1.16$\pm$0.25)$\times 10^{12}$ L$_{\odot}$ and the X-ray luminosity in the ROSAT energy band (0.1 - 2.4 erg\,s$^{-1}$) is 0.136$\times 10^{44}$\,erg\,s$^{-1}$. The Dressler \& Shectman Probablity that the cluster doesn't contain substructre is 0.24. This cluster is normal X-ray emitting cluster.\\\\
McDonald et al. (2011) presented the result of a combined X-ray and $H\alpha$ study of 10 galaxy and 17 galaxy clusters using Chandra X-ray Observatory and Maryland Magellan Tunable Filter. They find no difference in morphology or detection frequency of H$\alpha$ filament in group versus clusters over the mass range $10^{13}\,<$\,M$_{500}\,<10^{15}$\,M$_{\odot}$. The value of E(B-V) of A1139 is 0.031 and according to  Sun et al. (2009) the temperature of X-ray at $r_{2500}$, kT= 2.2.\\\\
Donzelli et al. (2011) derived detailed R-band luminosity profiles and structural parameters for a total of 430 brightest cluster galaxies, down to a limiting surface brightness of 24.5 mag\,arcsec$^{-2}$. Light profiles were fittled with Sersic's $R^{\frac{1}{n}}$ model also the logarithmic slope of the metric luminosity $\alpha$ was determined. The effective surface magnitude of A1139 is 21.39 and the effective radius is 6.74 kpc. The n parameter of this cluster is 4.38. The central surface magnitude is 22.45 and the scale length is 22.60 Kpc. The Sersic absolute magnitude and exponential absolute magnitude is -22.92 and -22.99   respectively. The total absolute magnitude is -23.71 whereas the ratio of Sersic to exponential magnitude is 0.93. The $\alpha$ parameter is 0.57. The inner and outer ellipticity is 0.06 and 0.32 respectively. The inner and outer position angle is -51.2 and -82.0 respectively. The metric absolute magnitude is -22.55.   \\\\
Hilton et al. (2013) has combined two data base ROSAT-ESO flux limited X-ray (REFLEX) galaxy cluster survey and 2dF Galaxy Redshift Survey (2dFGRS) survey to study the effect of the large scale cluster environment, as traced by X-ray luminoisity, on the properites of cluster member galaxies. $R_{200}$ calculated taking the $\sigma$= 550$\pm$50 km\,s$^{-1}$ and red shift 0.0396 is 1.3 Mpc. The X-ray luminosity of this cluster in 0.1-2.4 keV is 0.089$\pm$0.018$\times 10^{44}$\,erg\,s$^{-1}$. The completeness of this cluster is 0.90.\\\\
Burgett et al. (2008) carried out one, two and three dimensional tests for detecting the presence of substructure in clusters of galaxies obtained from the 2dF Galaxy Redshift Survey. The sample of 25 clusters used in this study includes 16 clusters not previously investigated for sub-structure. At least three clusters (A1139, A1663, AS333) exhibit velocity-position characteristics consistent with the presence of possible cluster rotation, shear or infall dynamics. The number of galaxies in A1139 is 109 and the proper distances $d_{p}(t_e)$=162 and $d_p(t_0)$=171 given in h$^{-1}$\,Mpc. The richness of A1139 according to Abell, Corwin \& Olowin 1989 is 0 and according to 2dFGR is less than 0. The analysis of A1139 done is based on 106 galaxies. The fitted optical core radius is 0.26 h$^{-1}$\,Mpc with central density of 138 which is fairly typical of the poor clusters in the sample. At large radii the ellipticity is low, $\sim$ 0.2 and the position angle is $\sim 65^{\circ}$. The inner part of this cluster is elongated and the ellipticity within 0.75 Mpc is $\approx$ $100^\circ$. The two dimensional test shows no asymmetry but three dimensional test indicates the presence of strong substructure. There exist a velocity gradient across the cluster. The shear seen in the velocity structure may indicate the cluster has a significant rotation. Alternatively, the cluster may be in the late stage of a double merger with two subclumps spiraling in to merge with the main system. The three dimensional tests and the cluster morphology indicate that the cluster is dynamically interesting. However Kriessler \& Beers (1997) and Krywult, MacGillivry \& Flin (1999) concluded that the cluster is unimodal with no substructure while West \& Bothun (1990) found marginal evidence of substructure.\\\\
Ebeling et al. (2008) present low-flux extension of the X-ray selected ROSAT Brightest Cluster Sample (BCS) published in  Paper I series. A1139 has $n_H$ 3.9 $\times 10^{20}$cm$^{-2}$ and the temperature 2.1 keV. The flux of the X-ray in range 0.1-2.4 keV is 4.2$\times 10^{-12}$\,erg\,cm$^{-2}$\,s$^{-1}$ and the luminosity is 0.29$\times 10^{44}$\,erg\,s$^{-1}$.\\\\
Struble and Rood (1999) presented a compilation of redshifts for 1572 Abell, Corwin \& Olowin cluster referenced to both heliocentric and cosmic background radiation reference frames, and 395 velocity dispersions corrected to the reference frame of the cluster. A1139 has the heliocentric redshift 0.0398  and the redshift with respect to cosmic background radiaion derived from $\bar{z_h}$ is 0.0386. The velocity disperson in the rest frame of the cluster is 351 km\,s$^{-1}$. The mass of A2162 is 12307.5 $\times 10^{12}$M$_{\odot}$  and the classical radius is 8. The velocity determined from CMB is 12079.8 km\,s$^{-1}$ but the observed velocity is 11730 km\,s$^{-1}$.\\\\
Krywult, MacGillivray \& Flin (1999) searched the presence of subclustering in 18 rich Abell cluster by the method based on the wavelet transform applied to the position of galaxies. A1139 is type I cluster and has redshift 0.0383 and contains 239 galaxies. Baier (1979, 1983), Baier \& Mai (1977, 1978), Bird (1993), Krywult et al. (1996, Surface number density contour plots), Krywult(1997, symmetry and separation test proposed by West et al. (1988)), West \& Bothun (1990) found no evidence of the subclustering in A1139. Also by applying wavelet analysis of the positional data for galaxies of 18 abell cluster Krywult et al. found A1139 are unimodal, show no evidence of any substructues.\\\\
Willick (1998) presented first result from a Tully-Fisher (TF) survey of the cluster galaxies. The galaxies are drawn from 15 abell clusters that lie in the redshift range $\sim$ 9000-12000\,km\,s$^{-1}$ and are distributed uniformly around celestial sky. A1139 is one of the sample cluster (Las Campanas Clusters) with redshift 0.383 and Abell richness 0 and photographic magnitude of the 10$^{th}$ brightest cluster member is 15.0 (ACO).\\\\
Dale et al. (2008) had obtained I band Tully-Fisher (TF) measurements for 522 late-type galaxies in the fields of 52 rich Abell clusters distributed throughout the sky between $\sim$ 50 and 200h$^{-1}$\,Mpc. They had estimated corrections to the data for various forms of observational bias, most notably cluster population incompleteness bias. The number of cluster members with reliable photometry and velocity widths in A1139 is 11. The cluster offsets from the template zero point $(a_{bias}-a_{tf})$ is -0.17.\\\\
Frithsch and Buchert (1998) discussed the implication of the fundamental plane parameters of clusters of galaxies derived from combined optical and X-ray data of a sample of 78  near by clusters. One of the cluster in their sample is A1139 whose optical luminosity is 123.110 $\times 10^{10}L_{\odot}$ and X-ray luminosity is 0.706 $\times 10^{44}$\,erg\,s$^{-1}$. The optical half-light radii is 1.138 Mpc.\\\\
Dale et al. (1999) presented first result of an all sky observing program designed to improve the quality of I band Tully-Fisher template and to obtain the reflex motion of the Local Group with respect to clusters to $z\sim 0.06$. They have used A1139 as a sample cluster and obtained the helocentric velocity 11855$\pm$59\,km\,s$^{-1}$ and velocity with respect to cosmicmicro backgroud is 12218 km\,s$^{-1}$. The value of V$_x$= -3026\,km\,s$^{-1}$,V$_y$= 10399\,km\,s$^{-1}$ and V$_z$= -5656\,km\,s$^{-1}$. \\\\
Batuski et al. (1991) presented result of fourteen North Galactic Cap Abell clusters that were previously unmeasured. Abell 1139 has distance class 3 and richness class 0. The velocity after eliminating the effect of the motion  of sun and earth is 11490\,km\,s$^{-1}$. The H, K, H$\beta$, Mg, Na lines were used to determine the redshift of this cluster.\\\\
Sandage et al. (1991) calculated $m(r)$ growth curve and the Petrosian $\eta(r)$ curve. A1139 has the redshift of 0.0376 and 1\,094 is the factor by which if we multiply the angular radius we calculate the linear radius in Kpc. f is derived from the redshifts and an assumed $H_0=50$\,km\,s$^{-1}$Mp$^{-1}$. The adopted redshift is from Schombert (1987). So the angular radius is $30.9^{\prime\prime}$ and linear radius is 33.8 Kpc. The total V magnitude from calculated and corrected m$(\infty)$ growth curve is 13.4 and adopted distance modulus from redshift is 36.77. The average effective surface brightness is 22.8 V\,mag\,arcsec$^{-2}$.\\\\
Plionis, Barrow and Frenk (1991) determined the shapes of the galaxy clusters found in projection. They used the Lick map of galaxies as their basic data set which enabled them to identify and analyze a sample clusters which is nearly an order of magnitude larger than those studied previously (Binggeli 1982). The mean ellipticity of Abell 1139 identified at 3.6 level is 0.207.\\\\
Huchra and Geller (1992) acquired redshifts for a complete sample of 351 abell cluster with tenth-ranked galaxy magnidute ($m_{10}$). They used this survey to search for large-scale coherent structures and the distribution of superclusters. Abell 1139 lies in richness class 0 and distance class 3. Its first rank magnitude is 13.80 and tenth rank magnitude is 15.00. The estimated red shift from Leir \& Bergh (1977) is 0.0420 and observed redshift is 0.0397.\\\\
Wise et al. (1993) examined the distribution of far-infrared emission from the cluster as a whole and it is clear from their observation that real flows must be inhomogenous, with a mixture of temperature and densities at given radius. The abell radius of A1139 is 49 arcmin and the common logarithm of X-ray luminosity is 43.47 erg\,s$^{-1}$. The cluster spiral fraction is 55\% and it belongs to class V (Low luminisity x-ray clusters). The IRAS infrared excess in $4^{\prime}$, 10$^{\prime}$, 20$^{\prime}$ and abell radius $r_a$ centered at clusted center in 60$\mu$m is 85, 45, 49, 38, 63 and in 100$\mu$m is 58, 52, 49, 37, 63 respectively.\\\\
Plionis (1994) estimated position angle of large number of Abell and Shectman (Shectman 1985) clusters, identified in the Lick map as surface galaxy-density enhancements. He also determined the major axis orientation of the 202 abell clusters. Abell 1139 also known as Shectman cluster 61 has the position angle 112.5, measured relative to North in the anti-clocksiwe direction and this defines the orientation of the cluster.\\\\
Wu and Han (1994) predicts that the effect of gravitational lensing by the matter associated with cluster of galaxies can magnify background, leading to an enhancement of source number density around foreground cluster of galaxies. They conduct  a search for the association of distant radio-bright quasars with Abell clusters using the 1-Jy and 2-Jy all sky catlogs. Abell 1139 is found to be near the quasar 1055+01 having redshift 0.888. The angular separation is $0.13^{\circ}$.\\\\