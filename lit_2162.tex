Giacintucci et al. (2013) presented the morphological study and spectral analysis for sample of 13 cD galaxies in rich and poor clusters of galaxies. Their study is based on new high sensitivity Gaint Meterwave Radio Telescope (GMRT) observations at 1.28 GHz, 610 MHz and 235 MHz. The GMRT full resolution image at 235 MHz shows two opposite lobes, with lack of a central compact component at both frequencies. The radio emission is weak and surface brightness is low. A similar double morphology without central component was found also at 1.4 GHz by Owen \& Ledlow (1997). The source has linear size of $\approx$ 90 $\times$ 38 Kpc. They also found that A2162 shows fairly relaxed morphology of the lobes and lack nuclear emission and these features are consistent with the idea that they are aged radio galaxies.\\\\
Liuzzo et al. (2010) presented new VLBI observations at 5GHz of a complete sample of Brightest Cluster Galaxies (BCGs) in Abell Clusters. The detailed discussion about the distribution of the absolute magnitude of BCGs which is not correlated to the dynamic equilibrium of their host-clusters can be found in Hoessel et al. (1980). These galaxies are intimately related to the collapse and formation of clusters: recent models suggest that BCGs must form earlier, and the galaxy merging with the clusters during collapse within a cosmological hierarchy is a visible scenario (Bernardi et al. 2006). The Abell cluster is a low X-ray luminosity clusters (Burns  et al. 1994). Its BCG is NGC6086 is a bright cD galaxy which hosts a double-lobed radio source. In the NVSS image, the total flux density is $\approx$ 108.7 nJy. The radio spectrum and morphology suggests that it is relic galaxy where the core radio activity stopped some time ago.\\\\
Abell 2162 is member of the Hercules super cluster complex (Einasto et al. 2001). The X-ray emission of this nearby cluster of galaxies has been studied by Ledlow et al. (2003) who reported an X-ray luminosity in the 0.5-2.0 keV band of 0.16 $\times 10^{43}$ erg\,s$^{-1}$ based on ROSAT data. The fading radio lobes of B2 1610+29 associated with NGC 6086, the brightest galaxy of the cluster located close to the X-ray peak.\\\\
Donzelli, Muriel, Madrid (2011) derived detailed R band luminosity profiles and light profiles were initially fitted with a Sersic's $R^{\frac{1}{n}}$ model. The effective surface magnitude and central surface magnitude is 20.85 and 22.46 respectively. The effective radius is 7.19 Kpc and n parameter is 3.92.\\\\
Gal et al. (2000) presents photometric redshifts for 431 Abell clusters imaged as part of the Palomar Abell Cluster Optical Survey (PACOS). For this cluster (\textit{g-r}) is 0.43867 and $g_{mean}$ = 20.2053.\\\\
Miller (2001) used the NRAO VAL Sky Survey (NVSS) to identify radio galaxies in eighteen nearby Abell clusters. A2162 contains only four radio galaxies consitent with the cluster at \textit{z} $\approx$ 0.03 whereas six radio galaxies are at \textit{z} $\approx$ 0. The zero point magnitude assumed for POSS II photometry of the cluster is 30.24 and average extinction assumed for POSS II photometry is 0.04. The separation between optical position and radio position as specified in NVSS catalog 29.2 and Radio flux at 1.4 GHz determined directly from the NVSS image 113.4.\\\\
Mahdavi et al. (2001) demonstrated that individual elliptical galaxies and clusters of galaxies form a continuous X-ray luminosity -velocity dispersion($L_X-\sigma$) relation. Their sample of 280 clusters have $L_x\propto\sigma^{4.4}$. For this cluster log$\sigma$ is 2.56$\pm$0.07 km\,s$^{-1}$ and log$L_x$ is 43.32$\pm$0.10 $h_{50}^{-2} $erg\,s$^{-1}$\\\\
Haynes et al. (2004) conducted a study of optical and HI properties of spiral galaxies (size, luminosity, H$\alpha$ flux distribution, circular velocity, HI gas mass) to explore the role of gas stripping as a driver of morphological evolution of clusters. The X-ray temperature of Abell 2162 is 0.9 keV and this is derived from the velocity dispersion. The bolometric luminosity of X-ray is 42.95 erg\,s$^{-1}$. The peculiar velocity, taken from Giovanelli et al. (1997) is 323 km\,s$^{-1}$.\\\\
Ruiter et al. (2008) presented a study of the optical brightness profiles of early type galaxies, using a number of a samples of a radio galaxies and optically selected elliptical galaxies. They collected data from Faber et al. and Laine et al. and found that the optical magnitude of Abell 2162 is -22.75 and  log(P$_t$/WHz$^{-1}$) at 1.4 GHz is 23.26.\\\\
Robinson et al. (2008) determined mass of A2162 is 520.0 M$_\odot$ and the classical radius is 8. The velocity determined from CMB is 9794.4 km\,s$^{-1}$ but the observed velocity is 9689 km\,s$^{-1}$.\\\\
Fritsch and Buchert (1998) discussed implication of the fundamental plane parameters of clusters of galaxies derived from combined optical and X-ray data of a sample of 78 nearby clusters. The optical luminosity of A2162 is 61.319 [$10^{10}L_\odot$] and X-ray luminosity is 0.091 [$10^{44}$\,erg\,s$^{-1}$]. The optical half-light radius of A2162 is 0.530 Mpc.\\\\
Carter et al. (2008) present kinematic parameters and absorption line strength for three brightest cluster galaxies. They have described A2162 as poor cluster or compact group.\\\\
Lauer and Postman (1994) measured the velocity of the Local Group with respect to an inertial frame defined by the 119 Abell and Abell, Corwin \& Olowin (ACO) clusters contained within 15000 km\,s$^{-1}$. Parameter subscript refer to the CMB frame (C), the local Group at rest with respect to the ACIF (L), and the frame implied by recovery of the local Group motion (L), with respect to the ACIF (F). For Abell 2162 $M_c=-22.594$, $M_l=-22.604$ and $M_F=-22.475$.\\\\
West (1989) generated a catlog of 48 probable supercluster  by means of percolation technique using a large sample of Abell clusters with measured redshifts \textit{z} $\leq$ 0.1 and various properties of these superclusters were examined. The orientation of clusters within these supercluster were also examined. Position angle data for clusters have been gathered from many sources in the literature and supplemented with new measurement presented in this paper. (SP=Struble and Peebles 1985, B=Bingeli 1982, D=Struble 1987 (he presents position angle determination for various surface brightness isophotes of first-ranked galxies); J=Tuker and Peterson 1988). The position angle of this abell cluster is 6, 0[SP], 1[B], 179[D], 7[J]. The final value predicted in this paper is 0.\\\\
Struble and Rood (1991) presented a list of redshifts for 758 Abell clusters and velocity dispersion for 121. They also presented another list of 33 Abell clusters with published redshifts, most of which are probably redshifts of foreground or background galaxies superposed on, or near the Abell clusters. In their paper the adopted $z_{adp}= 0.0320$ for this cluster and this was calculated using 5 galaxies. The velocity dispersion $\sigma_{adp}=323$ km\,s$^{-1}$.\\\\
Plionis, Barrow and Frenk (1991) identified a large number of galaxy clusters in the Lick map using an algorithm based on an overdensity criterion. The resulting catlogues contain $\approx$ 6000 clusters out of which 753 is Abell clusters. They determined ellipticities and position angle using Lick map. The mean ellipticity of Abell 2162 (identified at 3.6 level) is 0.301. The position angle determined from the Lick map is 1.7$\pm$10 using 3 overdensity thresholds to define the mean position angle. The value of position angle given by Sruble and Peebles (1985) is 0.\\\\
Henriksen (1992) used the available X-ray data to investigate systematic errors in a complete subsample of the Abell catalog which has been used in studies of large-scale structures. The catalogue of Rich Clusters of Galaxies identified by Abell (1958) contains 2712 entries and these cluster were classified by richness (0-5). The Abell 2162 was studied by Einstein telescope and found the luminosity to be $L_x=0.84\times 10^{43}$ erg\,s$^{-1}$. And the richness of this cluster is 0. It is BM type II-III and RS is Is.\\\\
Postman et al. (1992) have acquired redshifts for a complete sample of 351 Abell clusters with tenth-ranked galaxy magnitude ($m_{10}$) less than or equal to 16.5, including 115 entirely new cluster redshifts. The richness and distance class of Abell 2162 is 0 and 1 respectively. The $m_{10}$ is 13.70 and observed redshift of this cluster is 0.0318 for A2162.\\\\
Wise et al. (1993) had analyzed IRAS image data using a random position, multiple-aperture photometry method to study diffuse far-infrared emission for a sample of 56 clusters of galaxies at 60 and 100 $\mu$m. The radius of A2162 in arcminutes is 57 and log $L_x$=42.86 erg\,s$^{-1}$. This cluster is low-luminosity X-ray clusters.\\\\
Abell (1957) prepared catlogue of 2712 rich cluster of galaxies found on the National Geographic Society-Palmor Observatory Sky Survey. From this catalogue, 1682 clusters are selected to meet specific crietria for inclusion in a homogenous statistical sample. Some criteria to be clusters are richness criterion, compactness criterion, distance criterion, galactic latitude criterion. The richness group is from 0 to 5 which includes count of galaxies 30-49, 50-79, 80-129, 130-199, 200-299, $\geq 300$ respectively. The distance group interval is from 1-7 which inludes the tenth brightest member magnitude is 13.3-14.0, 14.1-14.8, 14.9-15.6, 15.7-16.4, 16.5-17.2, 17.3-18.0 and $\geq 18$. The magnitude of Abell 2162 is 13.7 , richness is 0 and distance is 1.\\\\
White (1978) presented the result of an extensive survey of a wide range of Abell and poor, non-Abell clusters of galaxies. The cluster are classified on a modified Bautz-Morgan system which includes detailed Yerkes form-types and relative brightness of the inner, brighter member galaxies of the cluster, and which is applicable to a wider variety of cluster and group of galaxies than the orginal Bautz-Morgan system. Abell 2162 has distance class 1 and richness class 0. It has $M_{10}=$13.8 and is BM type III. Its abell radius is 64 and central density estimate is I (Intermediate).\\\\
Fanti et al. (1981) carried out survey of 61 abell clusters included in the HEAO-2 satellite observing program was carried out at 1.4 GHz with the Westerbork Synthesis Radio Telescope. In this paper, they presented nine clusters of distance class 1 and 2 and make statistical consideration about the radio emission and structure of the cluster galaxies. Abell 2162 has distance class 1 and richness class 0. The heliocentric mean velocity of the cluster is 9585 km\,s$^{-1}$. It is BM type III and dominant galaxy type in this cluster is Spiral. In this cluster they identified three radio sources. This cluster belongs to a large supercluster including: A2197 and A2199 at $\delta\approx 40^\circ$, the Hercules completed at $\delta\approx 16^\circ$ and some other smaller galaxy groups at intermediate declinations (Ekers et al. 1978, Chincarini et al. 1981). It is at the center of a wide area where secondary consideration, Butcher and Oemler (1978) judge its inclusion in the Abell catalogue as ``a mistake". Out of six galaxies with measured velocity, two have v=15000\,km\,s$^{-1}$, indicating a contamination from back-ground galaxies. Also, evidence of the presence of hot gas in various locations within the supercluster has been pointed out by several author (Jaffe \& Perola 1974, Ekers et al. 1978, Valentijn 1979).