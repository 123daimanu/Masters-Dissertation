\section{Evolution of Galaxies in the Cluster}
\noindent Curosity and eagerness to understand why the nature is the way it is has gravitate human beings from ancient time to modern era. Among various whys and hows one of the most fundamental question that probes modern day physicst mind is why the universe is the way it is now. Placing beside the concept of multiverse, it is no doubt that universe is unique and it is out of question to do experiment and duplicate universe in lab, as done in other fields of physiscs, to study the various mechanism involved during its evolution. The only tools that cosmologist and astrophysist are equipped with are the data, theories and high performance computing environment to explore the mysteries hidden in the universe.\\\\
According to stardard model of cosmology, universe began along with the begining of space and time at a event called big bang, though most of the scientist think it is the misnomer. After big bang universe passed through series of events like inflation, leptogenesis, baryongenesis, radiation dominated phase, matter dominated phase and finally moving towards the dark energy dominated phase.\\\\
During these phases electrons, baryons and other particles were formed and density fluctations took place (originated from quantum fluctation during the inflation era). During radiation dominated era growth of structure was suppressed by tight interaction of photons and matter. Matter couldn't response to its own gravitaional force so density enhancement from the earlier time couldn't grow but at the time of recombination the density fluctations grew and latter formed large scale structure.\\\\
Basically, there are two models of formation of large scale structure.
\begin{itemize}
\item Top-Down model: In this model, large strucutre of galaxy clusters formed first and latter on fragmented into dwraf galaxies and merged to form large galaxies and galaxy clusters. This is based on the principle that radiation smoothed out the matter density fluctation to produce pancake. These pancake accrete matter after recombination and grow until they collapse and fragemnet into galaxies.
\item Bottom-up model: In this model, galaxy formed first and merged into clusters. The density enhancement at the time of recombination collapsed from self gravity into dwarf galaxies. And these attaract each other and merge to form larger galaxies and clusters.
\end{itemize}
One of the major part of understanding why the universe is the way it is now is to understand formation and evolution of large scale structures. And this can be studied with the help of information about the spin vectors of the galaxies and clusters.\\\\
\section{Literature Review on Galaxy Orientation Study}
Spin vector orientations of galaxies in the clusters can be an indicator of the initial conditions when galaxies and clusters formed provided that the angular momentum of galaxies haven't been altered too much since their formation. A useful properties of galaxies in cluster of which different theories make different prediction is the distribution of the angular momentum vectors.\\\\
%The angular momentum of galaxies in cluster and supercluster has been studied by numerous authors e.g Reihardt \& Robert 1972; Hawley \& Peebles 1975; Thompson 1976; Helou \& Salpeter 1982; Kapranidis \& Sullivan 1983; Djorgovski 1983, 1987; MacGillivray \& Dodd 1985; Flin \& Goldlowski 1986, 1989; Flin 1988; Hoffman et al 1989; Kashikaswa \& Okamum 1992 and many others.\\\\
Howley and Peeble (1975) proposed a method of studying the galaxy orientation. In their method, they observed position angle  and axial ratio distribution and also used the Fourier method of analysing position angle histograms. A second approach was proposed by Jaaniste and Saar (1978) and latter on modified by Flin and Godlowski (1986) to transform the 2-D projected data of image to 3-D information about the orientation.\\\\
The work on the orientation of spin vector of galaxies can be divided into three periods; before 1986, between 1986 and 2001 and after 2001. Each of the periods along with their findings are described in brief below.\\\\
Before 1986 several author used position angle distribution analysis to conclude the results about the galaxy alignment. Howley and Peeble investigated several areas and cluster but couldn't find anything interesting but Thomson (1976) studied  orientation of eight Abell clusters and found a tendency for galaxies in Coma to point toward the cluster center. Adams, Strom \& Strom (1980) studied seven Rood-Sastry L-type Abell clusters and found weak tendency of the galaxy's position angle to concentrate along the cluster position angle and perpendicular to it. They also found prominent alignment to Abell 2199, which was in agreement with Thomson's result. Macgillivary and Dodd (1979a, 1979b, 1985a and 1985b) investigated several groups and clusters and found weak tendency of galaxies to be aligned with, or perpendicular to, their radius vectors to the cluster or group center. Godwin, Metcalfe, and Peach (1983) published machine measurement of 6727 galaxies in the Coma region. Djorgovski (1983) analysed their data and observed very prominent alignment effects. Strom \& Strom (1978) studied the Perseus-Pisces Supercluster and noticed a distinct preference of the ellipticals to align with the main cluster chain. Gregory, Thompson, and Tifft (1981) noticed alignments and perpendicularity at a large scale in the Perseus-Pisces Supercluster. Reinhardt and Roberts (1972) found that spin vectors of local supercluster galaxies tend to lie perpendicular to the local supercluster plane. Jaanister \& Saar found opposite result. Macgillivary et al. (1982) found a marginally significant tendency for galaxies to be parallel to the local supercluster (LSC) plane with some secondary dependence on the supergalactic latitude. Kaprandis \& Sullivan (1983) noticed no significant effects. Macgillivary and Dodd (1985a) observed systematic effect both in alignmet and winding direction of local supercluster spirals, the effect being the strongest for the intermediate types. Dekel (1985) detected alignments in subsamples of elliptical galaxies. Helou (1984) carried out an interesting study of mutual orientations of the spin vectors in spiral pairs. He found that the spin vectors (SV) tend to be antiparallel, and the effect was stronger for the pairs with the lower mass-luminosity ratio.
In 1986 Godlowski \& Flin modified the approach of the Jaaniste and Saar and is known as position angle -inclination method or simply PA-inclination method. In this method two dimensional information is used to generate the full three dimensional information about the galaxy orientation. The result of the work published in the period 1986-2001 are given in table. \\\\%tabel ref
\begin{center}
\begin{tabular}[lp=7cm]{|l|l|l|l|}%\caption{Spin vector orientation study 1986-2001}
\hline\hline
Sample		 		& Selection criteria 			& Result and Discussion 					& Reference\\
\hline
1275 LSC				&Radial Velocity					&-anisotropic distribution 				& Flin \&\\ 
galaxies 			&(RV)$<$2600 					& -SVs of galaxies tend to lie			&Godlwoski\\	 
					&								&parallel to the plane						&1986\\
(UGC)				&km\,s$^{-1}$					&-anisotropy increases as axial 			&\\
					&								&ratio increases							&\\
 					& 								&-supports panckae model							&\\
 \hline
 618 LSC galaxies	&-RV$<$3000 km\,s$^{-1}$			&- isotropy in most cases				&Kashikawa\\
 (PANBG)				&- -3$\leq$ T					&- noticed bimodal distribution:			&\& Okamura\\
 					&								& galaxies near							&1992\\
 					&(morphological					& LSC plane lay their SVs parallel 		&\\	
 					&								&to , while 									&\\
 					&type index)						&those off the LSC perpendicular			&\\
 					&								& to the LSC 							&\\
 					&$\leq$10						&plane.										&\\
 					&								&-Virgo cluster member 					&\\
 					&								&show anisotropic 						&\\
 					&								& distribution, such that the 			&\\
 					&								&spin vectors							&\\
 					&								&tend to point towards Virgo 			&\\
 					&								&cluster center							&\\
 					&								&-supports pancake model					&\\
 \hline
 2227 LSC galaxies	&RV$<$2600 km s$^{-1}$			&-anisotropic distribution				&Godlwoski\\
(UGC,ESO,RCGR \&		&								&-SVs of galaxies tend to lie parallel 	&1993\\
TNGC)				&								& towards the Virgo center				&\\
					&								&-distribution of face-on and edge-on	&\\
					&								&galaxies are different.					&\\
					&								&-supports pancake model					&\\
\hline
2227 LSC galaxies	&RV$<$2600 km s$^{-1}$			&-anisotropic distribution				&Godlowski\\
(UGC,ESO,RCGR \&		&								&-azimuthal angle distribution 	&1994\\
TNGC)				&								& strongly depends on the 	&\\
					&								&supergalactic coordinate and RV									&\\
					&								&-supports pancake model.					&\\
					\hline

					
					

\end{tabular}
\end{center}
\begin{center}
\begin{tabular}[lp=7cm]{|l|l|l|l|}
\hline\hline
Sample		 		& Selection criteria 			& Result and Discussion 					& Reference\\
\hline
310 Virgo cluster 	&RV$<$2700 km s$^{-1}$			&-anisotropic bimodal distribution		&Hu et al.\\
galaxies (VCG, 		&								&-SVs projections of galaxies tend to	&1995\\
FGCP, UGC)			&								&point toward the Virgo cluster center	&\\
					&								&-noticed morphological dependence		&\\
					&								&-SVs of S and S0 galaxies tend to lie	&\\
					&								& perpendicular to the LSC plane			&\\
					&								&-SVs of S0+S galaxies tend to lie 		&\\
					&								&parallel to the LSC plane				&\\
					\hline
128 Coma 			&RV$\geq$4500					&-anisotropic distribution				&Wu et al.\\
cluster 				&km\,s$^{-1}$						&-morphologically dependent				&1997\\
(CCG,				&RV$\leq$9900 km\,s$^{-1}$		&-SV of S0 tend to lie parallel to the	&\\
CRCCG)				&m$_p\leq$15.6					&cluster plane							&\\
					&-R(radius from			&-SV of S tend to lie parallel or perp-	&\\
					&cluster center)					&endicular to the cluster plane			&\\
					&$\geq3^\circ$					&-SV projections of S tend to point 		&\\
					&								&towards the cluster center				&\\
					\hline
302 Field			&-RV$<$3000 km\,s$^{-1}$			&-anisotropy in the azimuthal angle		&Yuan et al.\\
galaxies				&- -3$\leq$T$\leq$10				&-projection of SV tend to point $\pm$30&1997\\
(PANBG)				&- R$\geq6^\circ$				&to the Virgo center.					&\\
					&								&-SVs of S galaxies tend to lie parallel	&\\
					&								&to the LSC plane						&\\
					\hline
220 Bright & RV$<$2500&-anisotropic distribution and&Hu et al.\\
isolated field&km\,s$^{-1}$&-morphologically dependent&\\
galaxies& &-supports pancake model for lenticular&\\
(PANBG)& &and hierarchy model for spirals&\\
\hline
491 galaxies&- b/a$<$0.30&-galaxy major plane tend to be& Godlowski\\
in the field of&(a \& b: major&perpendicular to direction of the PA of&\\
the Abell 0754&diameter &the major axis of the cluster&\\
(CUSSC)&respecitvely)&-angular momentum of the galaxies are&\\
 & &preferentally aligned parallel with the &\\
 & &cluster plane&\\
 \hline
 18 subcluster&RV$<$2800&-a strong systematic effect, generated by&Godlowski\\
 of the LSC&km\,s$^{-1}$&the process of deprojection of a galactic&\& Ostrosky,\\
 (TNGC,UGC,& &axis from its optical image, &1999\\
 LSCRC)& &is noticed in the catalog data.&\\
 \hline
557 galaxies&- 6000$\leq$RV&-anisotropic distribution&Flin, 2001\\
in the field of &$\leq$8000 km\,s$^{-1}$&galaxy rotation axes tend to lie in the&\\
Coma/A1367&- 11$^h.5<\alpha<$&main plane of the supercluster&\\
supercluster&13$^h.5,18^\circ$&-similar alignment found for ellipticals,&\\
(CGCG,UGC,&$<\alpha<32^\circ$&lenticular and spirals.&\\
RC3,PCG,NED)& &-projection of the rotation axes showed&\\
  & &preferential direction pointing towards the&\\
  & &supercluster&\\
  & &-supports pancake model&\\
  \hline
 
\end{tabular}
\end{center}
In 2001 Aryal \& Saurer developed a method to minimize the selection effect via numerical simulation. And after this most of the analysis were based on the PA-inclination method combined with the numerical simulation of the galaxies. The brief overview of the study of the spin vectors along with their result is given in the table. %reference to the table
\begin{center}
 \begin{tabular}[lp=7cm]{|l|l|l|l|}
\hline\hline
Sample		 		& Selection criteria 			& Result and Discussion 					& Reference\\
\hline
Coma cluster&6724 galaxies within&-overall anisotropic apperance &Kitzbichler\\
			&region of about&of galaxy ensemble on the smaller  &et al. 2003\\
			&2.6$^\circ\times2.6^\circ$&scale of substructure.&\\
			\hline
Abell 14&Galaxy with axial ratio&-major planes of galaxy&Baier et al.\\
	&less than &tend to be parallel to the&2003\\
	&0.75 &direction of PA of cluster.&\\
	& &-Angular momentum&\\
	& &perpendicular to the &\\
	&&cluster plane&\\
	\hline
Coma cluster		&ellipsities greater 	&no significant deviation 	&Torlina et al. \\
catalog by&than 0.1, 0.2, 0.3 &from isotropy &2007 \\
Eisenbardt &were examined & & \\
et al.(2007)& & & \\ 
\hline
Groups of &-no. of galaxies&-group major axis is& Godlwoski\\		
LSC from&at least 40 and&aligned to join of&\& Flin\\
NBG catlog&substructure contains&brightest galaxies&2010\\
(Tully&atleast 10 galaxis&-the acute angle between &\\
1988)&-total 61 groups&groups is not isotropic&\\
\hline
Data from &-cluster having atleast&-as richness increases&Godlwoski\\
PF catlog&100 galaxies and&isotropy increases&et al.\\
(Panko \& &having BM type&-Angular momentum increases&2010\\
Flin 2006)&taken. Sample&with mass&\\
&has 247 objects&Weak correlation of&\\
&&alignment and BM&\\
&&type&\\
\hline
PF catlog&-247 optically&-Anisotropy increases with&Flin et al.\\
(Panko \& &rich Abell clusters&richness of cluster&2011\\
Flin 2006)&containing more&-No connection of &\\
and Tully&than 100 galaxies&alignment with BM type&\\
NBG catlog&-b/a$<0.75$&&\\
(Tully 2008)&&&\\
\hline
PF catlog of&247 very rich clusters&-The anisotropy is greatly&Godlwoski \\
galaxy strucutre&having at least 100&increased when analyzed with&et al. 2011\\
(Panko \& Flin&members and identified in&respect to polar angle&\\
2006)&ACO clusters (Abell et al.)&than in position angle&\\
\hline
\end{tabular}
\end{center}
\begin{center}
 \begin{tabular}[lp=7cm]{|l|l|l|l|}
\hline\hline
Sample		 		& Selection criteria 			& Result and Discussion 					& Reference\\
\hline
Same as &-Axial ratio greater&-position angle distribution&Godlwoski\\
Godlowski&than 0.75&is isotropic&2012\\
et al. (2010)&-no significant difference&&\\
&&in Equatorial and&\\
&&Supergalactic coordinate&\\
&&systems&\\
\hline
Galaxies in LSC&-number of galaixes&-Random orientation&Pajowska\\
Group from Tully&in group more&-Weak alignment for&et al.\\
Nearby Galaxies&than 40&spiral galaxies&2012\\
(NBG)&&-Lack of alignment for&\\
&&less massive galaxies&\\
&&confirms alignment&\\
&&of galaxies increases&\\
&&with the mass&\\
\hline

\end{tabular}
\end{center}
Aryal \& Saurer (2004-2013) carried out a systematic study by analyzing huge database of POSSII and ESO survey and found few important results. They have used a similar data reduction procedure and methods. We summarize their results in a tabular form:
\begin{center}
\begin{tabular}[lp=7cm]{|l|l|l|l|}
\hline\hline
Database&Selection&Results&References\\
&criteria&&\\
\hline
Eight Abell& POSII \& ESO& -Abell 42, Abell 1775, Abell& Aryal \& \\
Clusters: BM&Film:1231&3558 and Abell 3560: supports&Saurer \\
type I&galaxies&pancake model&\\
&&-Abell 401, Abell 2199 and&\\
&&Abell 3566: supports Hierarchy&\\
&&model&\\
\hline
Local&UGC Catalog:	&-LSC spiral galaxies: supports&Aryal \&\\
Supercluster&4073&Pancake model&Saurer\\
Galaxies&morphologically&-Barred \& Irregulars: supports&2005a\\
&identified	&Hierarchy model&\\
&galaxies with&-Orientation of early-type and&\\
&radial velocity&late type galaxies are found&\\
&$<$3000&to be different&\\
&km\,s$^{-1}$&&\\
\hline
Several Abell	&POSSII\& ESO 	&-Abell 1412, Abell 2048 and	&Aryal \& Saurer\\
Clusters: BM		&Film:			&Abell 4038: supports Pancake& 2005b \\
type III			&851 galaxies	&model						&\\
&&-Abell 2061, Abell 2065, Abell	&\\
&&2151 and Abell 2187: supports&\\
&&Primordial Vorticity model&\\
\hline
\end{tabular}
\end{center}
\begin{center}
\begin{tabular}[lp=7cm]{|l|l|l|l|}
\hline\hline
Database&Selection&Results&References\\
&criteria&&\\
\hline

Galaxies in the		&ESO Film 1433&-no morphological dependence&Aryal \&\\
region $15^h28^m\leq$&galaxies&in the galaxy orientaion is&Saurer\\
$\alpha(2000)\leq$&&noticed&2005c\\
$19^h28^m,$&&-No preferred orientation can&\\
$-68^\circ\leq$&&be seen for the spiral galaxies&\\
$\delta(2000)$&&-major diameter of $<$47 arcsec&\\
$\leq -62^\circ$&&galaxies: supports Primordial&\\
&&Vorticity model&\\
\hline
Ten Abell&POSSII\& ESO&-Abell 1920, 2255 and 2256:&Aryal \&\\
Clusters: BM&Film:1315&bimodal orientation&Saurer \\
type II-III&galaxies&-Abell 268, 426, 1035, 1227&2006\\
&&1367 and 1904: supports&\\
&&Pancake model&\\
&&-Vanishing angular momentum&\\
&&for the richness class 0&\\
\hline
Shapley &ESO Film: 323&-No preferred alignments:&Aryal,\\
Supercluster&galaxies&supports Hierarchy model&Kandel \&\\
&&-Low radial-velocity galaxies in&Saurer\\
&&A3558 showed anisotropy&2006\\
\hline
Seven Abell&POSSII \& ESO&-A1767, A1809, A2554, A2721&Aryal,\\
Clusters: BM&Film: 786&supports Pancake model&Poudel \&\\
type II-III&galaxies&-noticed a systematic change&Saurer\\
&&(with distance, radial velocity&2007\\
&&morphology and magnitude) &\\
&&in the galaxy alignments from early-type&\\
&&(BM I) to late-type (BM II) clusters&\\
\hline
Nearby LSC&2228 UGC&-noticed a preferred spatial&Aryal, \\
galaxies&ESO galaxies:&alignment and non-chiral property&Paudel\\
&radial velocity&in the leading and trailing arm&\& Saurer\\
&3000-5000&spiral barred galaxies&2008\\
&km\,s$^{-1}$.&-predicts that the progressive&\\
&&loss of chirality might have some&\\
&&connection with the rotationally&\\
&&supported (spirals, barred spirals)&\\
&&and randomized (lenticulars,&\\
&&ellipticals)systems&\\
\hline
Nearby LSC&5169 UGC,&-spiral galaxies: supports&Aryal,\\
galaxies&ESO galaxies:&Pancake model&Neupane \&\\
&radial velocity&-early-type barred spiral&Saurer\\
&3000-5000&shows a random&2008\\
&km\,s$^{-1}$&orientation&\\
\hline
LSC+nearby&10562&-the galaxies that have radial&Aryal,\\
LSC galaxies&galaxies: radial&velocities 1500 to&Kafle \&\\
&velocity$<$&2000 and 3000&Saurer\\
&5000 km\,s$^{-1}$&to 3500 km\,s$^{-1}$ shows&2008\\
&&preferred alignment in&\\
&&both the two- and&\\
&&three dimensional analysis&\\
\hline
\end{tabular}
\end{center}
\begin{center}
\begin{tabular}[lp=7cm]{|l|l|l|l|}
\hline\hline
Database&Selection&Results&References\\
&criteria&&\\
\hline
35 clusters&POSSII \& ESO&-a dependence has been&Aryal,\\
&Films&noticed between the optical search&Bachchan \&\\
&&limit and the mean radial velocity&Saurer\\
&&of the cluster.&2010\\
&&-preferred position angle&\\
&&distribution of galaxies in the&\\
&&cluster has been found to be&\\
&&independent of the mean radial&\\
&&velocity of the clusters&\\
\hline
Field galaxies&1621 field&-random alignment is found in&Aryal 2011\\
&galaxies:&the PA-distribution of Z- and S-&\\
&radial&mode spirals&\\
&velocity&-a homogeneous distribution of the S- and&\\
&between&Z- shaped galaxies is found to&\\
&3000-&be invariant under global&\\
&5000&expansion&\\
&km\,s$^{-1}$&&\\
\hline
Three merging&merging binary&-A3395 that exhibits strongly&Aryal,\\
binary clusters&clusters A175&distorted feature in its maps of&Paudel \&\\
&A3395 and&temperature and surface brightness showed &Saurer\\
&A3528: SDSS&a random orientaion &2012\\
&database&-a relation between the stages&\\
&&of merging and the non-random&\\
&&alignments of galaxies in the&\\
&&substructure of the&\\
&&binary cluster is&\\
&&expected&\\
\hline
Zone-of-&POSSI prints:&-spin vectors of galaxies tend to&Aryal,\\
avoidance&410 galaxies&lie in the equatorial plane whereas&Yadav \&\\
galaxies&&these vectors tend to be oriented&Saurer\\
&&perpendicular to the Local&2012\\
&&Supercluster plane&\\
&&-random alignment of spin&\\
&&vectors of galaxies is noticed&\\
&&with respect to the&\\
&&galactic plane.&\\
\hline
\end{tabular}
\end{center}
Global rotation of galaxy clusters has been suggested for several galaxies (e.g Materne \& Hopp 1983; Oegerle \& Hill 1992, Dupke \& Bergman 2001), but there were still no conclusive evidence of cluster rotation. For the galaixes in cluster, the global velocity gradient suggestive of cluster rotation had been detected in several clusters (Materne \& Hopp 1983, Biviano et al. 1996, den Hortog \& Katgert 1996, Burgett et al. 2004). Huang and Lee used spectroscpic sample of galaxies in SDSS and 2dFGRS and found 12 tentative rotating clusters that have large velocity gradient out of 899 abell clusters. They have latter on selected six probable rotating clusters (A0954, A1139, A2162, A1399, A2169, A2366) that showed single density peak around the cluster center with a spatial segeration of high and low velocity galaxies. We have planned to work on three clusters of galaxies that have radial velocity less than 1600 km\,s$^{-1}$ (A1139, A2162, A2366).
\section{Motivations}
The motivation to carry out the present work are as follows:
\begin{itemize}
\item The distribution of the angular momentum vectors of galaxies in the clusters and Superclusters (gravitationally bounded system) can be an indicator of the initial conditions
when galaxies and clusters formed, provided the angular momentum of galaxies have
not significantly altered since their formation (Weizscker 1951 \& Gamow 1946). To
gain an idea of the origin of angular momentum of galaxies it is very important to
understand the evolution of large scale structures of the universe. Thus, a thorough
study of the spatial orientations of galaxies is needed.
\item Astrophysical database contains selection effects because of the limitations of all deep sky observation projects. The nature of the selection depends upon the types of telescope and the technology used for the noise reduction. Selection effects may play an important role when galaxies are taken from an incomplete
dataset. These effects can lead to artificial structures (Aryal \& Saurer 2001). In most
of the published works in galaxy orientation studies, selections of various kinds can be
seen. These are mainly due to the projection effects and inhomogeneity in positions.
Only few authors have made an attempt to minimize these effects (Godlowski
1994). It is essential to remove such effects to avoid misinterpretations in the results.
We intend to work on the removal of such effects.
\item Hwang and Lee (2007) proposed six rotating clusters by analysing SDSS and 2dFGRS database. We intend to study the angluar momentum distribution of galaxies in these clusters.
\end{itemize}
\section{Objectives}
Our objectives are as follows:
\begin{itemize}
\item As the spin vector orientation acts as the fossils of the large structure formation and evolution. Our main objective is to predict which model best explains the formation and evolution of large structure.
\item As we are concerned with a patch of galaxies (e.g. clusters), our study will help to know the local effects that plays a major role during the evolution of the galaxies.
\end{itemize}
